% !TEX TS-program = xelatex
% !TEX encoding = UTF-8 Unicode
% !Mode:: "TeX:UTF-8"

\documentclass{resume}
\usepackage{zh_CN-Adobefonts_external} % Simplified Chinese Support using external fonts (./fonts/zh_CN-Adobe/)
%\usepackage{zh_CN-Adobefonts_internal} % Simplified Chinese Support using system fonts
\usepackage{linespacing_fix} % disable extra space before next section
\usepackage{cite}
\geometry{%
	a4paper,
	left=0.2in,
	right=0.2in,
	top=0.4in,
	bottom=0.2in,
	nohead
} % 减少页边距

\begin{document}
\pagenumbering{gobble} % suppress displaying page number

\name{田亮}


\basicInfo{
  \email{liang.tian979@gmail.com} \textperiodcentered\
  \phone{(+86) 151-9594-2745} \textperiodcentered\
  \faGithub\ \href{https://github.com/fw-coder-tl}{https://github.com/fw-coder-tl}}

\section{\faGraduationCap\  教育背景 }
\datedsubsection{\textbf{河海大学}\quad 211 双一流 \hfill \textit{软件工程(本科)}}{2023-09 -- 2027-07}
语言能力: CET-4、CET-6

\section{\faCogs\ 专业技能}
% increase linespacing [parsep=0.5ex]
\begin{itemize}[parsep=0.5ex]
  \item \textbf{操作系统}:熟悉操作系统的相关理论,如内存管理、进程通信等,熟悉Select、Poll、Epoll等模型原理。
  \item \textbf{计算机网络}:熟悉计算机网络基础知识,如TCP连接、网络收包发包等底层原理、流量控制、拥塞控制等。
  \item \textbf{MySQL}:熟悉MySQL使用及原理,如索引、事务、锁、日志等,具有数据库以及SQL调优经验。
  \item \textbf{Redis}:熟悉Redis核心数据结构、持久化、数据过期、内存淘汰等原理,掌握如何解决缓存穿透、雪崩、击穿问题。
  \item \textbf{JVM}:熟悉JVM底层原理,如类加载机制、垃圾回收算法、JVM垃圾回收器等。
  \item \textbf{并发编程}:熟悉JUC并发工具的使用和原理,如线程池、锁等,以及深入理解JMM、线程安全知识。
  \item \textbf{Spring}:熟练使用SpringBoot进行Web后端开发,熟悉IOC、AOP等核心原理。
  \item \textbf{消息队列}:熟悉RocketMQ、Kafka消息中间件的使用,在项目中解决过重复消费、消息堆积等问题。
  \item \textbf{AI}:熟悉大语言模型应用开发,掌握Spring AI、RAG检索增强生成、MCP工具调用、Agent智能体架构设计。
  \item \textbf{设计模式}:深入理解责任链模式、策略模式、工厂模式等,能够灵活运用设计模式解决复杂业务问题。
\end{itemize}
\section{\faCode\ 项目经历}

\datedsubsection{\textbf{AI-Agent-Station}—基于多Client协作的智能体决策执行系统}{2025年6月 -- 2025年11月}
\textbf{项目描述:}面向企业级AI应用场景,设计并实现基于DDD架构的智能体决策执行平台。通过构建多个专业化Client(分析、执行、监督)实现任务分解、决策分析和精准执行,结合RAG知识库提供领域知识增强,解决传统AI应用在复杂任务处理中缺乏上下文记忆、无法多轮迭代优化的问题。已应用于智能客服、代码生成、数据分析等场景。\\
\textbf{技术栈:} \textit{Spring Boot、Spring AI、MCP、Redis、Elasticsearch、Kafka、MinIO}
\begin{itemize}
  \item 设计基于责任链模式的 Agent 执行链路架构,构建 Agent 完整链路,通过 Armory 装配服务动态加载不同角色的ChatClient,采用工厂模式 + 策略模式实现 Client 热更新,支持业务场景灵活组合,代码复用率显著提升。
  \item 集成RAG检索增强生成,设计RagAnswerAdvisor作为Advisor链前置处理器,在ChatClient调用前进行向量检索。通过Elasticsearch存储文档向量,结合KNN 向量召回与BM25重排序,实现“关键词+语义”的双引擎检索。
  \item 基于Kafka解耦文件上传、处理与向量化流程,实现文件分片上传与断点续传,实现系统削峰,单文档处理耗时从 5s 优化至 200ms。
  \item 实现流式响应(SSE)和动态上下文管理,通过ResponseBodyEmitter实时推送各阶段结果,集成SpringAIChat-Memory实现上下文持久化。通过流式输出将首响应时间从3s优化至0.5s,用户等待感知时间显著降低。
  \item 实现执行策略的可插拔设计,支持AutoAgent(自动多轮迭代)、FixedAgent(固定流程)、FlowAgent(流程编排)三种模式,通过策略模式实现动态选择,满足不同业务场景需求。
\end{itemize}

\datedsubsection{\textbf{GroupBuy-Market}—基于高并发秒杀拼团交易系统}{2024年10月 -- 2025年3月}
\textbf{项目描述:}基于DDD架构的拼团秒杀电商平台,支持高并发拼团、秒杀、优惠试算等核心业务。平台面向C端用户提供拼团购买、限时秒杀等营销活动,支持多商品、多活动并发场景。\\
\textbf{技术栈:} \textit{Spring Boot、MySQL、Redis、RocketMQ、XxlJob、ShardingSphere、ELK Stack}
\begin{itemize}
  \item 设计并实现责任链模式+规则引擎架构,通过工厂模式动态组装交易规则、结算规则、退单规则过滤链,支持规则树结构和规则可配置化,提升业务规则处理的灵活性和可维护性。
  \item 完成订单超时的数据处理,对订单数据进行分库分表、冷热分离,提高数据库的读写性能。
  \item 基于Redis pub/sub模式,结合自定义注解和BeanPostProcessor实现轻量级配置中心,支持限流阈值、降级开关、人群黑白名单的热更新,实现了生产环境的无感发布与动态切量。
  \item 参考JDHotKey开源项目,设计并实现HotKey热点探测与动态路由系统:采用Filter拦截器+本地缓存(Cafeine)+Redis二级缓存架构,实现热点商品毫秒级实时探测,通过规则匹配引擎和异步上报机制实现热点识别;设计动态路由策略,根据商品热度自动路由到秒杀/普通拼团接口。
  \item 设计并实现高并发秒杀库存扣减方案:采用Redis预扣减+MQ事务消息异步处理+MySQL乐观锁四层保障,实现库存秒杀QPS从100提升至700+,解决超卖问题。设计旁路检测机制和流水对账补偿任务,通过XxlJob定时任务对比Redis与MySQL库存差异,自动修复数据不一致问题,有效应对网络抖动、服务重启、消息丢失等异常场景。
\end{itemize}




% Reference Test
%\datedsubsection{\textbf{Paper Title\cite{zaharia2012resilient}}}{May. 2015}
%An xxx optimized for xxx\cite{verma2015large}
%\begin{itemize}
%  \item main contribution
%\end{itemize}


%% Reference
%\newpage
%\bibliographystyle{IEEETran}
%\bibliography{mycite}
\end{document}
